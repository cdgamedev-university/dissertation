\documentclass[journal]{IEEEtran}
\usepackage{amsmath,amsfonts}
\usepackage{algorithmic}
\usepackage{xcolor}
\usepackage{array}
\usepackage[caption=false,font=normalsize,labelfont=sf,textfont=sf]{subfig}
\usepackage{textcomp}
\usepackage{stfloats}
\usepackage{url}
\usepackage{verbatim}
\usepackage{graphicx}
\graphicspath{{./figures/}}
\hyphenation{op-tical net-works semi-conduc-tor IEEE-Xplore}
\def\BibTeX{{\rm B\kern-.05em{\sc i\kern-.025em b}\kern-.08em
    T\kern-.1667em\lower.7ex\hbox{E}\kern-.125emX}}
\usepackage{balance}
\usepackage{biblatex}
\usepackage{listings}
\usepackage{float}
\usepackage{hyperref}

\definecolor{codegreen}{rgb}{0,0.6,0}
\definecolor{codegray}{rgb}{0.5,0.5,0.5}
\definecolor{codepurple}{rgb}{0.58,0,0.82}
\definecolor{backcolour}{rgb}{0.95,0.95,0.92}

\lstdefinestyle{mystyle}{
    backgroundcolor=\color{backcolour},   
    commentstyle=\color{codegreen},
    keywordstyle=\color{magenta},
    numberstyle=\tiny\color{codegray},
    stringstyle=\color{codepurple},
    basicstyle=\ttfamily\footnotesize,
    breakatwhitespace=false,         
    breaklines=true,                 
    captionpos=b,                    
    keepspaces=true,                 
    numbers=left,                    
    numbersep=5pt,                  
    showspaces=false,                
    showstringspaces=false,
    showtabs=false,                  
    tabsize=2
}

\lstset{style=mystyle}

\addbibresource{references.bib}

\begin{document}
\title{The automated detection of fraudulent peer-to-peer transactions in massively multiplayer online economies}
\author{Hayley Davies - 1902055}

\maketitle
\footnote{Source code for all algorithms is available on GitHub: \url{https://github.com/cdgamedev/dissertation}}
\footnote{Source \LaTeX is available on Falmouth GitHub: \url{https://github.falmouth.ac.uk/Games-Academy-Student-Work-22-23/1902055-comp3xx-dissertation}}

\begin{abstract}
Massively Multiplayer Online games are a hugely popular and successful subsection of the gaming industry. These games allow players to trade items within the game, but some players choose to buy and sell in-game items for real-world money, known as Real Money Trading. This leads to people preferring to buy from illicit sources rather than through the game itself, resulting in a loss of revenue for the developers. Additionally, this practice enables people to make money through methods such as botting, account theft, or cheating.
\end{abstract}

\begin{IEEEkeywords}
Anomaly Detection, Massively Multiplayer Online, Real Money Trading
\end{IEEEkeywords}

\section{Introduction}
\IEEEPARstart{R}{eal-Money} Trading (RMT) is the practice of buying and selling virtual items and currency within massively multiplayer online (MMO) video games. This practice often violates the games' Terms of Service or Code of Conduct\cite{AmazonGamesCOC}\cite{SquareEnixCOC}, and as a result, players who engage in RMT are often at risk of being banned from the game. RMT has negative effects on the game's economy and community, and it is therefore discouraged by game developers and players alike. The solution to this issue is anomaly detection algorithms. These are useful in serveral facets of computer science and this paper will discuss the use of these algorithms and compare different anomaly detection algorithms which utilise Standard Deviation.

\section{Literature Review}
\subsection{Anomaly Detection for RMT}
\noindent Anomaly detection is a technique used to help address the issue of RMT in MMOs\cite{Tao2019}\cite{Ahmad2009}. By using computers to flag in-game transactions automatically for human review, it is possible to identify and prevent RMT activity in a way where humans don't have to analyse the all the data manually by hand. Preventing RMT helps to protect the game's economy and maintain a fair and balanced playing experience for all players. However, it is important to note that the effectiveness of this approach will depend on the specific details of the game and the methods used for detecting anomalies.

Fujita et al. propose a method for addressing the issue of RMT in MMO games, which involves identifying suspects, verifying their involvement in RMT activity, and banning their accounts\cite{Fujita2011}. They also classify RMT players into three categories: 
\begin{itemize}
    \item \textit{Sellers} are those who sell the virtual property to players for real-world money.
    \item \textit{Earners} acquire virtual property (currency and items) from non-player characters (NPCs) and real players.
    \item \textit{Collectors} convey virtual property from earners to sellers.
\end{itemize}

Fujita et al. manually classified a set of players and then used an algorithm proposed by Newman and Girvan\cite{Girvan2002} to extract communities and further identify players. They ranked players based on the number of times they traded currency, the number of trades they made in total, and the total volume of currency traded.

Fujita et al. argue that RMT is harmful to both the game's economy and its players, as it is often associated with other illicit activities such as cheating, botting, and account theft. RMT can also drive away legitimate players who become frustrated with the problems it causes, and it can discourage new players from joining the game\cite{Sifa2021}. Overall, Fujita et al. believe that RMT is a serious issue that needs to be addressed to protect the integrity and health of MMO games. Han et al. and Sifa et al. back up Fujita's claims adding that "cheating in MMOs often reduces the shelf life of the game" by causing people to abandon it\cite{Han2022}\cite{Sifa2021}.

\subsection{Types of Anomaly}
\noindent Anomalous data can typically be categorized in three different ways\cite{Ahmed2016}\cite{Bhuyan2013}. These categories are:
\begin{itemize}
    \item \textbf{Point Anomaly} where a data point is unusually out of range.
    \item \textbf{Contextual Anomaly} (or collective anomaly\cite{Song2007}) where sometimes, a data point which seems anomalous is actually within range depending on a varying factors.
    \item \textbf{Collective Anomaly} is where multiple data points are out of range, but when considered individually, are not out of range.
\end{itemize}
Understanding these categories can help researchers identify and address potential issues in their data. By detecting anomalies, it may be possible to uncover hidden patterns or trends and improve the accuracy and reliability of data-driven systems and models.

\subsection{Machine Learning}
\noindent Due to the nature of anomaly detection is often done with Machine Learning (ML) algorithms\cite{Omar2013}\cite{Misra2020}\cite{Zong2018}. These algorithms are split into two categories, supervised and unsupervised\cite{Omar2013}. Supervised methods "require a labeled training set containing both normal and anomalous samples", whereas, unsupervised methods don't require training data as they assume a fraction of data points are anomalous\cite{Omar2013}.

Nassif et al. state that they "recommend that researchers conduct more research on ML studies of anomaly detection to gain more evidence on ML model performance and efficiency" following their own review of prior research. They mention that unsupervised datasets have a greater number of research papers than supervised datasets. They also identify 29 different machine learning models for detecting anomalies\cite{Nassif2021}.

\subsection{Nearest-neighbor Based Algorithms}
\noindent Local Outlier Factor (LOF) works by getting each object in the dataset to "indicate its [own] degree of outlier-ness"\cite{Breunig2000}.

k-Nearest Neighbor (kNN) is a supervised learning algorithm based on Nearest Neighbor and it's discussed frequently within anomaly detection and ties in with lazy learning algorithms. kNN functions by finding the K number of the nearest points and assigns the data point to a label that is most suited, this can be used for anomaly detection if the assignment is different to how the data is already labelled\cite{Cover1967}.

Lazy learning algorithms work by\cite{Wettschereck1997}:
\begin{itemize}
    \item Storing all training data, and deferring processing until queries are given the required replies.
    \item Answering queries by combining the training data.
    \item After replying, the answer and any results are discarded.
\end{itemize}

Many improvements to the base kNN algorithm have been developed by other researchers. Muti-label kNN (ML-kNN) exists to provide lazy learning to problems such as text categorization or bioinformatics. This approach works by \cite{Zhang2005}.

\subsection{Graph-based Anomaly Detection}
\noindent Graph-based anomaly detection algorithms "[detect] patterns (substructures) within graphs" where a "substructure is a connected subgraph in the overall graph"\cite{Noble2003}. Noble et al. utilized their anomaly detection, Subdue, for intrusion detection, utilizing data which "contained 41 features describing the connection" and labelling them as "one of 37 different attack types" or "normal". This is a form of unsupverised learning\cite{Noble2003}. Graph-based anomaly detection works well with large datasets and is often used for collective anomalies\cite{Egilmez2014}.

Davis et al. discuss the use of Yet Another Graph-based Anomaly Detection Algoritm (YAGADA). They find that in other methods of anomaly detection, single time events are detected easily, but anomalous patterns of data which are anomalous in context such as "an airport technician who regularly hangs around in the baggage handling area, or a clerk who is spending an unusually long time on their own
in the cash room" would not normally be detected. They conclude that YAGADA works well for static graphs and suggest using it in "forensic analysis of graph transaction databases". They also conclude that LOF\cite{Breunig2000} is more suitable to numeric anomaly detection.

\subsection{Autoencoders}
\noindent Misra et al. focus on an autoencoder based model for detecting fraudulent transactions within the financial domain, specifically credit cards\cite{Misra2020}. They propose a two stage method where "a lower dimension of features are extracted from the input" before "a model decides whether the transaction is fraud or not". The first stage utilizes an autoencoder and the final stage utilizes a classification algorithm. "Autoencoders are simple learning circuits which aim to transform inputs into outputs with the least possible amount of distortion"\cite{Baldi2011}. They state that having too many features can cause classification algorithms to "run poorly" and that the "data becomes very expensive [when] time complexity is concerned" and is resolved by reducing the number of features. Misra defined features or attributes as parts of a whole data point and that autoencoders can extract these features nicely on any dataset. For credit card fraud, some of these attributes are, time/amount/mode/location of transaction, a user's account number, a user's age.

Deep Autoencoding Gaussian Mixture Model (DAGMM) works by preserving "information of an input sample in a low-dimensional space" and then performs a "Gaussian Mixture Model over the learned low-dimensional space" before utilizing "a sub-network called estimation network that takes the low-dimensional input from the compression network and outputs mixture membership prediction for each sample"\cite{Zong2018}.

\subsection{Standard Deviation}
\noindent Yang et al. discuss the use of standard deviation (SD) within anomaly detection\cite{Yang2019}. However, they note that simple datasets can cause false positives\cite{Pollet2017} and researchers misuse SD methods frequently\cite{Simmons2016}. The main issue with SD algorithms is that outliers influence the standard deviation and averages for the dataset.

To solve these issues Yang set out to develop a modified SD algorithm which could be relied on to give more accurate results. Two-stage thresholding (2T)\cite{Yang2019} works similarly to Clever Standard Deviation (Clever SD)\cite{Buzzi2011} by utilizing recursion. 2T works by recursively removing outliers one at a time and is the most accurate method of SD algorithms based on Yang's findings.

\subsection{Anomaly Detection for Other Datasets}
\noindent Bergman and Hoshen talk about anomaly detection for general data and classification of anomalies using AI. Examples, of where this is used, are for fraudulent credit transactions and detecting cyber attacks amongst others\cite{Bergman2020}. They state that "classification-based methods have dominated supervised anomaly detection", these are methods of anomaly detection which utilise a classifier trained by an ML model. They further discuss the use of the following semi-supervised methods; one-class classification and geometric-transformation classification. They make a comparison between SVMs\cite{Schoelkopf1999}, LOF\cite{Breunig2000} and DAGMM\cite{Zong2018}.

Similarly, Misra and Sadineni investigate the use of anomaly detection within Credit Card transactions\cite{Misra2020}\cite{Sadineni2020}.

\section{Research Questions}
\noindent The questions this paper sets out to answer are:
\begin{itemize}
    \item Which anomaly detection algorithm is the most performant for peer-to-peer transactions within MMOs?
    \item Can most fraudulent transactions be found using anomaly detection algorithms?
\end{itemize}

By answering these, game developers working on MMO titles can easily identify and choose a method that suits the needs for their game. This will help reduce the revenue loss caused by RMT which could be millions\cite{Dibbell2007}.

\section{Hypotheses}
\subsection{Which anomaly detection algorithm is the most performant for peer-to-peer transactions within MMOs?}
\noindent Following research conducted, for data specifically from Lost Ark, a simple method which doesn't require training data is most likely to be best. The 2T algorithm\cite{Yang2019} could be beneficial for a dataset which has a small number of features as the data from Lost Ark requires only 2 features, a more complex algorithm likely isn't required.

\subsection{Can most fraudulent transactions be found using anomaly detection algorithms?}
\noindent Due to the fact that the dataset used cannot be labelled, it will be difficult to quantify if the algorithms function to detect all fraudulent transactions in the dataset. If a data point is out of range then the algorithm, assuming the correct settings, will be able to correctly identify all anomalous transactions. However, not all fraudulent transactions are anomalous and therefore, legitimate transactions could be detected as fraudulent. RMTers could also realise that selling a 5 Gold Gem for 100,000 Gold is unreasonable, and therefore they can adapt strategies and sell thousands of Gems for 8 or 9 Gold and still profit substatially.

\section{Computing Artefact}
\subsection{Game Background}
\noindent For this research, the primary focus will be on Smilegate and Amazon Games' Fantasy MMORPG; Lost Ark\cite{LostArk2019} with a specific focus on its Gem system. Gems are collected by players during gameplay and can be sold to others using the in-game marketplace. Each Gem has different attributes, such as, Level, Name, Tier, Gem Effect, Sale Price and Sale Date.
\begin{itemize}
    \item \textit{Level} is a number between 1 and 10 which impact the effect of the Gem. Level \( n \) gems are created by merging 3 Level  \( n-1 \) gems. A Level 10 gem should always cost more than a Level 1 gem as it takes \( 3^{10} \) Level 1 gems to make a single Level 10 gem.
    \item \textit{Name} is a tag given to the gem and doesn't effect the gem. This means that name shouldn't affect the price.
    \item \textit{Tier} is a number between 1 and 3 for all items in Lost Ark, however, gems only exist at the start of Tier 2 meaning all gems will either be tagged with Tier 2 or Tier 3. Tier 3 gems get unlocked at Item Level 1302 and offer higher effects than their Tier 2 counterparts. This means that a Level 1 Tier 2 gem should cost less than a Level 1 Tier 3 gem, however, a Level 10 Tier 2 gem is likely to cost more than a Level 1 Tier 3 gem.
    \item \textit{Gem Effect} is the overall effect of the gem. This will either, increase the damage or decrease the cooldown of an ability in the game. Various gem effects will be more favourable based on the specific character build a player chooses. Favourable gem choices depend on the games meta and players often utilize a service like Maxroll\cite{Maxroll2022} to get the best build for their characters class. Gem Effect can have an impact on the price of a gem, however, this data is near impossible to get without direct access to the internal marketplace database. However, gems can also be rerolled for silver which means that there shouldn't be a huge price disparity between two Level 1 or two Level 10 gems of the same tier.
    \item \textit{Sale Price} is the price which a gem was sold for.
    \item \textit{Sale Date} is the date which a gem sold on.
\end{itemize}

For this research, an anomaly detection algorithm would utilize 4 factors; Level, Tier, Sale Price and Sale Date. This should be enough for a basis to detect prices which are too high at any given time. It also allows us to investigate fluctuations in price of gems over time which could be due to various factors in the game, such as events which inflate the number of gems within the game through increased drop rates of gems, gem giveaways or a change in the number of players which leads to less supply/demand.

\subsection{Detection Algorithms}
For this paper, 4 different algorithms were written which can detect anomalies. These revolve around checking each data point and if the deviance of that data point is within a specific range. All results from these algorithms, use a constant threshold which scales to the remaining dataset. This initial threshold has not been tailored per algorithm.

\subsubsection{Two Stage Thresholding Algorithm}
Figure \ref{algorithm:2t} showcases the 2T algorithm written for use within this paper. This algorithm is recursive and follows the general function outlined by Yang\cite{Yang2019} in their original algorithm. This algorithm works similarly to the Mean Standard Deviation with its main change being that it runs recursively.

\subsubsection{Clever SD Algorithm}
Figure \ref{algorithm:clever-sd} showcases the Clever SD algorithm written for use within this paper. This algorithm is recursive and follows the general function outlined by Buzzi\cite{Buzzi2011} in their original algorithm. This algorithm removes a single anomaly per function call until all anomalies have been removed.

\subsubsection{Mean Standard Deviation Algorithm}
Figure \ref{algorithm:sd-mean} showcases a function for standard deviaion around the mean. This is a common approach and checks if all datapoints are in range. If they are outside of the range, they are added to the anomalies array and the anomalies array is returned after the function is finished.

\subsubsection{Median Standard Deviation Algorithm}
Figure \ref{algorithm:sd-median} showcases a function for standard deviaion around the median. This is a common approach and checks if all datapoints are in range. If they are outside of the range, they are added to the anomalies array and the anomalies array is returned after the function is finished.

\section{Data Collection Methodology}
\begin{figure*}[!htbp]
    \centering
    \includegraphics[width=\textwidth]{figures/auction-house-before.png}
    \caption{Auction house screenshot before image processing.}
    \label{figure:auction-house-before}
\end{figure*}
NOTE: SOME DATA FOR 2022.11.03 IS MISSING DUE TO FILE CORRUPTION.

\subsection{Internal Data}
\noindent The best method of gathering data, is to contact the developers directly. The developer's internal policy could impact the ability to get the raw data from them directly. They may log a lot more data than is publicly accessible via the marketplace, for example, the internal data could contain account identifying information.

\subsection{Public Data}
\noindent In Korea, a public resource exist which allows users to view trasactional data across the entire auction house\cite{LostArkStove2019}. However, this website is inaccessible without name and age verification, due Korea's Game Industry Promotion Act\cite{GIPA2020kr}.

\subsection{In-game Data}
\noindent Another method for data collection surrounds screenshotting pages from the Sale History of Gems from Lost Ark's Marketplace, as seen in Figure \ref{figure:auction-house-before}. Then, the screenshot is cropped to remove the Gem Name, Starting Bid and Quality fields and converted to a greyscale image along with other image manipulation techniques to ensure clear text, shown in Figure \ref{figure:auction-house-after} (see Figure \ref{algorithm:image-formatter}).

Optical character recognition (OCR) is then used on the image and checked automatically for any errors (see Figure \ref{algorithm:image-processor}). The data can then be randomly sampled and manually reviewed to ensure the accuracy of the data following this process. Due to the nature in which this algorithm works, unreadable data will be logged in the saved file as a "-" alongside its page and entry numbers, making the issue much quicker to rectify.

\begin{figure}[H]
    \centering
    \includegraphics[scale=0.25]{auction-house-after}
    \caption{Auction house screenshot after image processing.}
    \label{figure:auction-house-after}
\end{figure}

Removing the name is done as this has no bearing on the price of a gem. Removing the Starting Bid field is done because the Starting Bid is optional when adding a gem to the marketplace, it is also not indicative of RMT. Instead, for the RMT transaction to occur correctly, Gems have a Buy Now Price set to a specific value. Removing the Quality field is done since gems don't have a quality value; this is always "-".

A consideration which could also affect the price of Gems is the "Gem Effect". This is a specific ability that Gem does to impact a character's Damage or Cooldown Time on a specific move. However, the Gem Effect is optionally rerolled within the game player. Rerolling requires the use of Silver, a much easier resource to gather, so this doesn't have a huge impact on the price of the gem.

Using this data, humans can easily recognise when a sale price for a specific level/tier gem is too high compared to other gems being sold at around the same time.

\section{Validation and Verification}
\noindent Without having a labelled dataset, it will be difficult to accurately verify anomalous transactions using an algorithm. However, by utilising different algorithms for the testing of data, it is possible to check overlapping anomalies between different algorithms. This will allow for the verification that a data point is anomalous.

The algorithms can also be validated by running them multiple times. As computation is likely to be nearly instantaneous running on modern machines, running it hundereds of times, the time to run the algorithms can be calculated more accurately. To ensure accuracy and authenticity

\subsection{Verifying Algorithms by Comparing Results of Different Algorithms}
\noindent Comparing the results is a way in which the data can be validated. By assigning each gem with a unique value based on its location within the database we can check each gem for its occurrence as an anomaly across multiple algorithms. An example in Python would look like \ref{algorithm:algorithm-comparison}

This method, however, suffers from a problem where anomalies which aren't detected by any algorithm won't be verified at all, leading to a reduced accuracy rate.

\section{Considerations}
\subsection{Legal}
\noindent Smilegate, the developers of Lost Ark, didn't explicitly give permission for their dataset to use be used in this paper, however, it doesn't breach their Code of Concuct\cite{AmazonGamesCOC}. This dataset is also available

\subsection{Ethical}
\noindent The data processed in the paper is gathered directly from Lost Ark's public marketplace where Smilegate likely already follow data anomyization practices\cite{Gruschka2018}. The data gathered doesn't contain identifiable information and therefore conforms to both General Data Protection Regulation (GDPR) and the Nuremberg Code\cite{Nuremberg1947}. As this papers research is carried out at Falmouth University, it also follows the Research Policy\cite{FalmouthEthicsPolicy2022}.

\subsection{Professional}
\noindent Professionally, it's important that this paper follows the BCS Code of Conduct\cite{BCSCodeOfConduct2022}. This includes working for the public interest by outlining solutions to current problems rather than ways to make current problems worse.

\section{Results and Analysis}
\noindent Figures \ref{graph:gem-datasize}, \ref{graph:anomalies-per-algorithm} and \ref{graph:time-taken-per-algorithm} showcase the results found from running the aforementioned algorithms, 2T, Clever SD, Mean SD and Median SD. In these graphs, Gem Number is used as a quick comparital measure between two merged properies each gem has. A Tier 2 Level 1 Gem is labelled as Gem Number 1 and a Tier 3 Level 1 Gem is labelled as Gem Number 11. Each gem tier has 10 levels, meaning, a Tier 2 Level 5 Gem is Gem Number 5 and a Tier 3 Level 5 Gem is Gem Number 15.

The data in Figure \ref{graph:gem-datasize} showcases the total size of the datasets - or gems sold - throughout the period 25th August 2022 until 29th December 2022. Overall, 14928 transactions were recorded for this paper. The trends shown in this graph help quantify the data shown in \ref{graph:anomalies-per-algorithm}.



\begin{figure}[h]
    \centering
    \includegraphics[scale=0.25]{figures/graph-gem-datasize.png}
    \caption{Graph showing the dataset size for each gem type.}
    \label{graph:gem-datasize}
\end{figure}

\begin{figure}[h]
    \centering
    \includegraphics[scale=0.25]{figures/graph-per-algorithm.png}
    \caption{Graph showing the number of anomalies found for each gem depending upon the algorithm.}
    \label{graph:anomalies-per-algorithm}
\end{figure}

\begin{figure}[h]
    \centering
    \includegraphics[scale=0.25]{figures/graph-time-taken-per-element.png}
    \caption{Graph showing the average computation time for different algorithms on each dataset.}
    \label{graph:time-taken-per-algorithm}
\end{figure}

\section{Disccussion}
\noindent [INSERT SECTION]

\section{References Section}
\printbibliography

\clearpage
\begin{appendices}
    \section{Data Collection Code Samples}
    \label{appendix:collection-code}
    
    \lstinputlisting[language=python]{algorithms/compare-algorithms.py}
    \begin{figure}[hbt!]
        \caption{Comparison Algorithm.}
        \label{algorithm:algorithm-comparison}
    \end{figure}

    \lstinputlisting[language=python]{algorithms/clever_sd.py}
    \begin{figure}[hbt!]
        \caption{CleverSD Algorithm.}
        \label{algorithm:clever-sd}
    \end{figure}
    
    \lstinputlisting[language=python]{algorithms/twostage_sd.py}
    \begin{figure}[hbt!]
        \caption{2T Algorithm.}
        \label{algorithm:2t}
    \end{figure}
    
    \lstinputlisting[language=python]{algorithms/mean_sd.py}
    \begin{figure}[hbt!]
        \caption{Standard Deviation (Mean) Algorithm.}
        \label{algorithm:sd-mean}
    \end{figure}
    
    \lstinputlisting[language=python]{algorithms/median_sd.py}
    \begin{figure}[hbt!]
        \caption{Standard Deviation (Median) Algorithm.}
        \label{algorithm:sd-median}
    \end{figure}

    \lstinputlisting[language=python]{algorithms/image-formatter.py}
    \begin{figure}[hbt!]
        \caption{Image Formatter Algorithm.}
        \label{algorithm:image-formatter}
    \end{figure}

    \lstinputlisting[language=python]{algorithms/image-processor.py}
    \begin{figure}[hbt!]
        \caption{Image Processor Algorithm.}
        \label{algorithm:image-processor}
    \end{figure}
\end{appendices}

\end{document}