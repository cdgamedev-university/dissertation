\documentclass{article}

\usepackage[style=numeric, sorting=none, defernumbers=true]{biblatex}
\addbibresource{references.bib}

\title{Comparing methods of automatically detecting fraudulent peer-to-peer transactions in massively multiplayer online economies.}
\author{Hayley Davies - 1902055}
\date{2022}

\begin{document}

\begin{titlepage}
        \maketitle
\end{titlepage}

\section{Introduction}
Real-Money Trading (RMT) is the method of buying and selling virtual items and currency within massively multiplayer online (MMO) video games. It often violate the games' Terms of Service or Code of Conduct and is therefore often a bannable offense within the respective game.\cite{AmazonGamesCOC}\cite{SquareEnixCOC}

Fujita et al. state that RMT is a harmful practive to both the game's economy and players due to it revolving around other illicit activity such as cheating, botting or stealing user accounts. They continue to state that this also impacts the game by pushing away legitimate players who become frustrated with the problems and can cause a lack new players based on what they have heard.\cite{Fujita2011}

\section{Literature Review}
Due to the nature of RMT, it requires detection and intervention to ensure that the economic security of the game is safe. Fujita et al. suggest that the way RMT should be handled is through identification of suspects, verification that the suspect is participating in RMT and then banning the account. They also state that there are multiple classifications of players within RMT; Sellers, Earners and Collectors.\cite{Fujita2011} Fujita et al. manually classified a set of players before proceeding to extract communities using a Newman and Girvan's proposed algorithm. This then allowed them to further identify players and rank them utilising the number of times the user traded currency, the number of times they traded in total, and the total volume of currency they traded.

Detecting RMT requires the use of anomaly detection. Ahmed et al. discuss the types of anomalies and suggest there are three different types of anomalies. Point Anomaly where a data point is unusually out of range. Contextual Anomaly where sometimes, a data point which seems out of range is actually within range depending on a varying factors. Collective Anomaly is where multiple data points are out of range, but when a single one of those data points exists on its own, its not out of range.\cite{Ahmed2016}



\section{Body}

\section{Professional Considerations}

\section{Conclusion}

\section{Acknowledgements}

\section{References}

\section{Reflective Addendum}

\section{Appendices}

\medskip

\printbibliography

\end{document}